%&LaTeX   -*-LaTeX-*-
% $Id$

\documentclass{article}
\usepackage{amsmath,amssymb,amsfonts}
\usepackage{mathbbol}
\usepackage[latin1]{inputenc}
\usepackage{float}
\usepackage{graphicx}
%\usepackage{algorithms}
\usepackage[noend]{algpseudocode}
\usepackage{cite}
\usepackage{hyperref}
\usepackage{tikz}

\usepackage{fancybox}

\graphicspath{{figs/}}

\newcommand{\bigOh}{\mathop{\mathcal{O}}\nolimits}  % big O
\newcommand{\esimo}[1]{\text{${#1}$-th}}              %n-th
\newcommand{\node}[2]{\langle {#1},{#2}\rangle}
\newcommand{\Prob}[1]{\Pr\left\{{#1}\right\}}
\newcommand{\Exp}[1]{\mathop{\mathbb{E}}\left\{{#1}\right\}}
\newcommand{\Var}[1]{\mathop{\mathbb{V}}\left\{{#1}\right\}}
\newcommand{\email}[1]{E-mail: \texttt{#1}}
\newcommand{\hypergeom}[6]{{}_{#1}{#2}_{#3}           %
\left(\left.\genfrac{}{}{0pt}{}{#4}{#5}\,\right|\, #6\right)} % hipergeometrica
\newcommand{\Ind}[1]{\mathbb{1}_{{#1}}}
\newcommand{\eqlaw}{\stackrel{\mathcal{D}}{=}} 
\newcommand{\C}{\mathcal{C}}
\newcommand{\U}{\mathcal{U}}
\newcommand{\K}{\mathcal{K}}



\newtheorem{theorem}{Theorem}
\newtheorem{defn}{Definition}

\renewcommand\algorithmiccomment[1]{\(\qquad\triangleright\) #1}%
\newcommand{\Proc}[1]{\text{\textsc{#1}}}
\newcommand{\To}{\textbf{\ to\ }}
\newcommand{\swap}{\leftrightarrow}
\newcommand{\call}[2]{\textsc{#1}({#2})}
\floatstyle{ruled}
\newfloat{algorithm}{thp}{loa}
\floatname{algorithm}{Algorithm}

\hypersetup{%
colorlinks = true,
breaklinks = true,
linkcolor = blue,
anchorcolor = blue,
citecolor = magenta,
pagecolor = blue,
urlcolor = red,
filecolor = blue,
bookmarksopen = true,
bookmarksnumbered = true,
backref = true,
}

\newsavebox{\mybox}
\newlength{\mylength}

\newenvironment{comment}[1][\linewidth]{
\setlength{\mylength}{#1}
\begin{lrbox}{\mybox}\begin{minipage}{\mylength}
\color{red}}{
\end{minipage}\end{lrbox}
\bigskip\centerline{\shadowbox{\usebox{\mybox}}}}

\title{Lottery Sampling, Lottery Saving and other Algorithms
  for Top $k$ Frequent Elements}

\author{Conrado Mart{\'\i}nez \and Gonzalo Solera} %[2]

\date{\today}

\begin{document}
\maketitle
\begin{comment}
  Es muy buena idea que vayas escribiendo en este documento de trabajo
  u otro equivalente que constituya la base de tu documentaci�n del
  TFG, las diferentes cosas que vayas haciendo, de la manera m�s
  organizada que puedas, aunque no es� muy pulido. Este documento te
  da un buen punto de partida sobre c�mo organizar tu TFG (habr�s
  de incluir algunas secciones ``chorras'', you know it) y tambi�n
  va estableciendo el marco de trabajo (p.e. esquema de nombres
  para los diferentes algoritmos, la plantilla ``unificada'' para todos ellos,
  las definiciones de las medidas de rendimiento, \ldots)
\end{comment}

\section{Preliminaries}
Consider a (large) data stream $\mathcal{Z}=z_1,\ldots,z_N$, where each $z_j$ is
drawn from some domain or \emph{universe} $\mathcal{U}$, and let $n\le N$ 
be the number of distinct elements in $\mathcal{Z}$. We may thus look at 
the multiset $X$ underlying $\mathcal{Z}$
\[
X = \{x_1^{f_1},\ldots,x_n^{f_n}\}
\]
where $x_1$, \ldots, $x_n$ are the $n$ distinct elements that occur 
in $\mathcal{Z}$ and $f_i$ denotes the number of occurrences (absolute 
frequency) of $x_i$ in $\mathcal{Z}$.
We will assume, w.l.o.g., 
that we index the elements in $X$ in non-increasing order
of frequency, thus $f_1\ge f_2 \ge \cdots \ge f_{n-1} \ge f_n > 0$.
We will use $p_i = f_i/N$ to denote the relative frequency of $x_i$.
For simplicity, we will assume that $f_1 > f_2 > \cdots > f_n$ in the
definitions below---they can be more or less easily adapted to cope
with elements
of identical frequency.

The two problems that we want to study here are:
\begin{enumerate}
\item Top $k$ most frequent elements. Given $\mathcal{Z}$ and a value
$k\le n$, we want to find 
$\{x_1,\ldots,x_k\}$ (or any subset of $k$ distinct elements
  with maximal frequencies). 
\item Heavy hitters. Given $\mathcal{Z}$ and a value $c$, $0 < c < 1$, 
we want to find (or count) the number
of distinct elements in $\mathcal{Z}$ with relative frequency $p_i \ge c$.
Those elements are called \emph{heavy hitters}. 
Given the data stream and the value $c$, we want to obtain
$\{x_1,\ldots,x_{k^\ast}\}$, where $k^\ast$ is
the largest value $k$ such that $p_k \ge c$. The value $k^\ast$
is the number of heavy hitters. 
\end{enumerate}
Moreover, in both problems, we might want that the algorithm
  returns the frequency $f_i$ of the returned elements. 

  None of these two problems can be solved exactly unless we can keep
  $\Theta(n)$ elements in memory; thus under the tight memory
  constraints of the data stream model, we must aim at approximate
  good solutions. Hence, the algorithms that we describe next might
  return elements which are not among the most frequent elements, or
  that are not heavy hitters, and rather than the frequencies $f_i$ of
  the returned elements, we will have to content ourselves with
  estimations $f'_i$ of the real frequencies.

We will concentrate in algorithms for
top $k$ most frequent elements. Notice that there can be at most
$\lfloor 1/c\rfloor$ heavy hitters in a data stream, and thus an algorithm
that retrieves the top $k^\ast=\lfloor 1/c\rfloor$ most frequent elements will
obtain all the heavy hitters.


\section{The Algorithms}
The algorithms that we shall consider next fail under the following
scheme.  They all keep a \emph{sample} $\mathcal{S}$ of up to $m$
distinct elements, for some value $m\ge k$ fixed in advance. Each element
$x$ in the sample is eqquiped with a frequency counter
\[
f(x) = f_\text{obs}(x) + f_\text{ini}(x),
\]
which is the sum of two components: $f_\text{obs}(x)$ is the number of
times that $x$ has been observed in the data stream since it has been included
in $\mathcal{S}$; $f_\text{ini}(x)$ is an estimation of the number of
occurrences of $x$ prior to the occurrence in which $x$ has been added
to the sample.

In the initial phase, the algorithms populate the sample with the
initial distinct elements in the data stream, recording their
frequencies. In particular, $f_\text{ini}(x)=0$ for these elements,
and $f(x)$ will be the real frequenciy of $x$ in the ``prefix'' of
the data stream examined so far.  This first phase ends when $m$
distinct elements have already been collected and an \esimo{(m+1)}
distinct element occurs in the data stream, or when the data stream is
exhausted---because $m\ge n$.  In the latter case, the sample has
complete information about the full data stream and all required
information can be obtained exactly.  In the interesting case, when $m
< n$ (typically, $m\ll n$), the algorithms loop through the remaining
items in the data stream $\mathcal{Z}$. For every incoming item $z$,
if $z$ is already in $\mathcal{S}$, the algorithms update $f(z)$ and
perhaps perform some additional bookkeeping (depending on the
particular algorithm under consideration), but not much more has to be
done. On the other hand, if $z\not\in\mathcal{S}$ then we apply some
criterium $\texttt{add?}(z,\mathcal{S})$;
if $\texttt{add?}(z,\mathcal{S})$ is not satisfied,
then $z$ is discarded. Otherwise, some element $z^\ast\in\mathcal{S}$ is choosen
and evicted from the sample ($z^\ast=\texttt{element\_to\_evict}(\mathcal{S})$),
and $z$ is added to $\mathcal{S}$; the frequency counter is initialized
$f(z)=1+f_\text{ini}(z)$. As in the case where $z\in\mathcal{S}$, some
additional bookkeeping might be necessary for each particular algorithm.

The algorithms that we will consider differ
hence three main aspects:  
\begin{enumerate}
\item When do we add to the sample an incoming item $z\not\in\mathcal{S}$
  ($\texttt{add?}(z,\mathcal{S})$? We will consider algorithms that
  make the decision by first generating a number $t\in[0,1]$,
  a \emph{token},
  uniformly and independently at random. If $t \ge t^\ast$ then
  $z$ will be added, where $t^\ast\in[0,1]$ is a \emph{threshold} value defined
  by each particular algorithm, and in general will depend on the ``evolution''
  of the data stream---namely, on the state of the sample and all
  the tokens generated so far\footnote{The well-known SpaceSaving algorithm
    might be seen as a particular instance of this scheme with $t^\ast=0$;
    therefore the generation of tokens is not necessary at all.}. 

  The details about how $t^\ast$ is computed---this includes
  whether tokens are generated
  for every instance in the data stream, or only for those for which we need to
  decide whether we add or not them to the sample---will lead to several
  different ``subfamilies'' of the generic algorithm.
  
\item Which element $z^\ast$ do we evict from the sample when a new element
  $z$ has to be added to the sample
  ($\texttt{element\_to\_evict}(\mathcal{S})$)?
  Again, this choice might be probabilistic.
\item How do we estimate the frequency $f_\text{ini}(z)$, that is, the number
  of prior occurrences of a new element $z$ that is to be added to the sample?
\end{enumerate}

Take for instance the well-known \textsc{SpaceSaving} (SS) \cite{SpaceSaving}.
It always
adds incoming items $z\not\in\mathcal{S}$, that is,
$\texttt{add?}(z,\mathcal{S})=\textbf{true}$. The element $z^\ast$ selected
for eviction is one with lowest estimated frequency
\[
z^\ast = z_\text{lef} = \text{arg min}_{x\in\mathcal{S}}\{f(x)\}.
\]
And the newly added item $z$ inherits the frequency of the evicted item:
$f_\text{ini}(z)=f(z^\ast)$.

A first simple variation of SpaceSaving that we will call
\textsc{Randomized SpaceSaving} (RSS) will behave as SS, but the
addition of the current elements $z$, if not in $\mathcal{S}$,
is decided a threshold $t^\ast>0$. The simplest case is
to have a fixed threshold, say $t^\ast=0.95$, but more sophisticated
schemes are also possible if the threshold $t^\ast$ is dynamically computed,
for example, $t^\ast$ might be the mean value of the 
tokens providing ``entrance'' to the $m$ current elements in the sample.

Generalizing the ideas above we move on a family of algorithms
where the main distinctive feature is that for every element in the sample
we keep a \emph{ticket} $t(z)$ which is the maximum among the
tokens generated for $z$. While in RSS variants
we do not generate a token when the current element $z$ in
the data stream is already in the sample, in the algorithms that we will
explore below tokens are generated for each element of the data stream,
whether they already belong or not to the sample.

Let us know consider the first algorithm in this family, that we call
\textsc{LotterySampling} (LS).
Each element $x$ in the sample has, besides its frequency counter, a ticket
$t(x)\in[0,1)$. When a new item $z$ in the data stream is retrieved
LS generates, uniformly at random in $(0,1)$, a random nuber $t$ for $z$. If $z$
was already in the sample its frequency counter $f(z)$ is updated (as usual),
but also its ticket: if $t > t(z)$ then $t(z):= t$; in other words,
$t(z)$ is always the largest random number generated for $z$. If $z$ is not
in the sample, we compare $t$ with the minimum ticket
$t_\text{min}=t(z_\text{min})$
in the sample. If $t > t_\text{min}$ then $z$ is added to the sample,
otherwise $z$ is discarded. The element $z^\ast$ to be evicted
is $z_\text{min}$, the one with the
minimum ticket, and the frequency counter of $z$ is initialized with
\[
f_\text{ini}(z)=\left\lfloor\frac{1}{1-t_\text{min}}\right\rfloor.
\]
Notice that when $z$ is added to the sample $t$ must be the largest
random number generated for $z$ so far, hence, we also set $t(z):=t$.
We will actually do that ($t(z):=t$) whenever an element $z$ enters
the sample in all the variants that we consider below. However, in
some of them, it is not true that $t(z)$ is the largest random number
ever generated for $z$, and hence, it's not, rigorously speaking, the
\emph{ticket} of $t$.  The property is true in those variants in which
the minimum ticket $t_\text{min}$ is always
evicted when a new element $z$ enters the sample, either because we evict
$z_\text{min}$ or becasue we swap tickets between $z\_text{min}$ and $z^\ast$
(\emph{ticket stealing}).

\textsc{LotterySaving-LowestEstimatedFrequency} (LS-LEF) combines some
of the ideas of the two algorithms above. When an item $z$ is not in
the sample we compare the random number $t$ with the ticket
$t_\text{lef}=t(z_\text{lef})$ of an element $z_\text{lef}$ with
smallest frequency counter (in case of ties, any of them is picked at
random). Notice that $t_\text{lef}$ might be or not the minimum ticket
in the sample. If $t > t_\text{lef}$ then $z$ is added and
$z^\ast=z_\text{lef}$ is evicted, and $f_\text{ini}(z)=f(z^\ast)$. 

\textsc{LotterySaving-LeastRecentlyObserved} (LS-LRO) is similar to LS-LEF,
but the ticket
$t$ of a candidate element $z$ to enter the sample is compared with the
ticket $t_\text{lro}=t(z_\text{lro})$ of the item $z_\text{lro}$
in the sample that was observed least recently (i.e., 
$z_\text{lro}$ is the item with frequency counter updated the longest ago). 
Like in LS-LEF, if $t > t_\text{lro}$ then $z$ is added and
$z^\ast=z_\text{lro}$ is evicted. The counter of $z$ is initialized using
$f_\text{ini}(z)=f(z_\text{lro})$.

In \textsc{LotterySaving-SmallestTicket} (LS-ST) the element
to be evicted is $z_\text{min}$; but instead of the rule
for initialization of $f$ of LS it uses the estimated frequency $f$ of $z^\ast$,
like LS-LEF and LS-LRO. That is, when an item $z$ is not in the sample
we compare its
ticket $t$ with the smallest ticket $t_\text{min}=t(z_\text{min})$ of the
element $z_\text{min}$
with smallest ticket (w.l.o.g. we can safely assume that no two tickets
ever are identical).
If $t > t_\text{min}$ then $z$ is added
and $z^\ast=z_\text{min}$ is evicted, like in LS. But, unlike LS,
we have $f_\text{ini}(z)=f(z^\ast)$.


\textsc{LotterySaving-Threshold} (LS-THR-$\theta$) has a parameter
$\theta\in[0,1]$.
We omit the parameter $\theta$ in the name, thus write LS-THR in
generic discussions about
this algorithm.
A new item $z\not\in\mathcal{S}$ is added to the sample if and only if
the ticket $t$ generated for $z$ is larger that $\theta\cdot t_\text{max}$,
where $t_\text{max}$ is the largest ticket among the elements in the sample
(LS-THR-0 is equivalent to SS).

If $z$ is added to $\mathcal{S}$, LS-THR evicts the element $z^\ast=z_\text{lef}$
with minimum frequency counter and $z$ inherits the frequency counter
of $z^\ast$ as the initial estimation: $f_\text{ini}(z)=f(z^\ast)$.

\textsc{LotterySaving-AboveMean} (LS-AM) adds a  new item to the sample
if its ticket $t$ is larger that the mean value $\overline{t}$ of all
the tickets in the sample. When $t > \overline{t}$, the evicted item $z^\ast$
is $z_\text{lef}$, one with smallest frequency counter,
and $z$ inherits $f(z^\ast)$ like
in LS-LEF and LS-THR.

Another variation is \textsc{LotterySaving-AboveMedian} (LS-MED) which adds
a new item to the sample if its ticket $t$ is above the median $t_\text{med}$
of the tickets in the sample. The remaining choices for the algorithm are as
in LS-AM. LS-MED can be generalized to consider the $\alpha$-quantile
(LS-QUANT-$\alpha$) of the tickets in the sample. Notice that LS-MED is 
LS-QUANT-0.5; the algorithm uses $t_{(\lceil \alpha\cdot m\rceil)}$ to decide
whether to include or not a new item in the sample, where $t_{(r)}$
denotes the $\esimo{r}$ smallest ticket in the sample, $1\le r\le m$.

The basic variants of LS-LEF and LS-LRO can be modified to incorporate
the ideas behind LS-AM and LS-MED.
Thus LS-LEF-AM and LS-LRO-AM will add an element
$z\not\in\mathcal{S}$ to the sample if $t > t(z^\ast)$
\textbf{or} $t$ is above the mean value $\overline{t}$ of the tickets
in the sample. The evicted item $z^\ast$ and the initialization of
the frequency counter of $z$ is as in the corresponding basic algorithms.

Instead of the mean value $\overline{t}$ of the tickets in the sample,
we might consider the median of the tickets, or more generlly, the
$\alpha$-quantile of the tickets, getting the variants LS-LEF-MED
LS-LRO-MED, or more generally, LS-LEF-QUANT-$\alpha$ and LS-LRO-QUANT-$\alpha$. 

Notice that it makes no sense to ``combine''
LS-AM and LS-QUANT with LS-ST since $t_\text{min}<\overline{t}$ and
$t_\text{min}\le t_{(r)}$ for all $r$.

Another source of variations stems from the way tickets are updated
when an item $z$ already in the sample is the incoming element from
the stream.  In all cases above we have assumed that tickets are
updated via $t(z):= \max(t, t(z))$, where $t$ is the ticket generated
for the current instance of $z$, whereas $t(z)$ is the ticket
associated to $z$ in the sample.  We can derive corresponding variants
of all the others discussed above were the items retain the ticket $t$
with which they entered the sample, and never update it once in the
sample, for instance.

All algorithms using tickets can be also characterized as defining a
\emph{thresold value} $t^\ast$ and a candidate item $z^\ast$ for eviction. The
criterium for addition of a element $z\not\in\mathcal{S}$ is always whether
$t > t^\ast$ or not, where $t$ is the ticket that was generated for
the current instance.
For example, lottery sampling (LS) takes
$t^\ast\equiv t_\text{min} = \min\{t(z)\,|\,z\in\mathcal{S}\}$,
and $z^\ast$ an element with the
minimum ticket $t_\text{min}$. Recall that we can safely assume that
all tickets in $\mathcal{S}$ are distinct.

With these conventions the template for all the algorithms discussed above is
as follows:
\begin{verbatim}
Populate S with the first m distinct elements in Z, 
         updating tickets and frequencies
while Z is not exhausted do
      z:= current item in Z
      t:= Random(0,1)
      if z in S then
           Update the ticket t(z)
           f(z):= f(z) + 1
      else if t > t* then
           S:= S - {z*} + {z}
           t(z):= t
           f(z):= fini(z) + 1
      else // do nothing
      Update t* and z*
endwhile
Report the k elements in S with largest f'
\end{verbatim}

Table \ref{table:algorithms} summarizes the characteristic values of
$t^\ast$ and $z^\ast$. Observe that we have considered two options for
the initialization of $f_\text{ini}(z)$ and three options for $z^\ast$
($z_\text{min}=\text{the element with minimum $t$}$,
$z_\text{lef}=\text{an
element with minimum $f$}$, and $z_\text{lro}=\text{the
least recently observed element}$); if we contemplate $C$ different ways
to define $t^\ast$ ($C=9$ in our table), we could consider then up to $6C$
different possible combinations---some make no sense.
If we add on top of that the choice
to update or not tickets of sampled elements we raise the count to
$12C$. Moreover, LS-THR and LS-QUANT depend on an additional real parameter
in $[0,1]$ giving us an additional source of variability.

\begin{table}
  \begin{tabular}{llll}
    Algorithm & $t^\ast$ & $z^\ast$ & $f_ini(z)$ \\\hline\hline
    SS        & 0       & $z_\text{lef}$ & $f(z^\ast)$ \\
    LS        & $t_\text{min}=t(z^\ast)$ & $z_\text{min}$
    & $\lfloor\frac{1}{1-t^\ast}\rfloor$ \\
    LS-LEF    & $t(z^\ast)$ & $z_\text{lef}$ & $f(z^\ast)$ \\
    LS-LRO    & $t(z^\ast)$ & $z_\text{lro}$ & $f(z^\ast)$ \\
    LS-ST    & $t(z^\ast)$ & $z_\text{min}$ & $f(z^\ast)$ \\
    LS-THR    & $\theta\cdot t_{(m)}=\theta\cdot t_\text{max}$ & $z_\text{lef}$ & $f(z^\ast)$ \\
    LS-AM     & $\overline{t}=\sum_{z\in\mathcal{S}}t(z)/m$ & $z_\text{lef}$ & $f(z^\ast)$ \\
    LS-QUANT  & $t_{\lceil\alpha m\rceil}$ & $z_\text{lef}$ & $f(z^\ast)$ \\
    LS-LEF-AM & $\min\{\overline{t},t(z^\ast)\}$ & $z_\text{lef}$
    & $f(z^\ast)$ \\
    LS-LEF-MED & $\min\{t_{\lfloor(m+1)/2\rfloor},t(z^\ast)\}$ &
    $z_\text{lef}$ & $f(z^\ast)$
  \end{tabular}
\end{table}

The three strategies LS-ST, LS-LEF and LS-LRO are almost identical, they
only differ in the definition of the element $z^\ast$ to be evicted.
These three strategies will be called \emph{canonical}, as all the
other algorithms that we have discussed here can be seen as variantions of one
of the three; maybe in the way that tickets are updated, maybe in the way
that $t^\ast$ is defined (e.g., $t^\ast\not=t(z^\ast)$), or the way that
$f$ is initialized (e.g., $f_ini(z)\not=f(z^\ast)$).

It is clear that the combinations that we can make ``playing'' around
with the criteria for addition ($\texttt{add?}(z,\mathcal{S})$), the
criteria for eviction and the initialiation of the frequency counter
are almost endless, but here we will restrict ourselves to the those
that we have been able to analyze and for which the experiments give
the best results. Another important issue in our choice has been the
efficiency with which the different operations can be supported.
Thus for instance, to implement LS-LEF it is very convenient to
maintain the elements of $\mathcal{S}$ sorted by frequency counter; locating
the least frequent element becomes trivial, and maintaining the order of the
sample after each frequency update is also very easy and efficient, as
they increment one by one.

\begin{comment}
  Gonzalo: agrega a esta seccion otros algoritmos que estes probando.
  En el documento final es posible que hayas de hacer una ``purga'',
  eliminando las opciones que no han dado buenos resultados. Sin
  embargo, SpaceSaving, Lottery Sampling y las tres variantes
  canonicas LS-ST, LS-LEF y LS-LRO deben estar siempre, como
  referencia. Para nuevas variantes inventa nombres y acronimos. Tambien
  habriamos de pensar algun modo facil de indicar si una estrategia es con o sin
  update de tickets, en vez de inventar nuevos nombres. Por ejemplo, LS*-LEF es
  la variante de LS-LEF sin hacer update de los tickets.

  Por cierto que nn una primera impresion yo
  diria que LS*-LEF es casi lo mismo que
  SpaceSaving pero lanzando una moneda e incluyendo al item z nuevo en
  el sample con probabilidad $m/(m+1)$ (con probabilidad $1/(m+1)$ el item
  nuevo no reemplaza a $z_\text{lef}$). 
\end{comment}

\subsection{Implementing the Algorithms}

\section{Measures of Quality}
Of course, for a top $k$ frequent elements algorithm we would like
that it reports the true $k$ most frequent elements and their
respective frequencies, but we know that this will not be possible unless
we had a linear amount of memory ($Theta(n)$)---this includes
the trivial situation when $m\ge n$. Hence we will assume in waht follows
that $m\ll n$, and we would like that
our algorithms return good approximations: elements that are
the most frequent or close to that, and good estimations of their respective
frequencies.
Likewise, when the task at hand is to report heavy hitters, i.e., elements
with relative frequency above a given threshold $c$, we might not report
some heavy hitters (or even report non-heavy hitters as if they were;
this might happen for algorithms working with overestimations
of the real frequencies).

In order to measure the quality of our algorithms and other contenders, we will
introduce several metrics. The first two metrics which we will introduce
are inspired by the very well known
\emph{recall} and \emph{precision} from Information Retrieval (IR), and
we will use the same names. These two metrics measure the quality of
the answer to a query to an IR system. Some of the documents retrieved
by the system will not be ``relevant'' to the user, whereas some
of the relevant documents will not be retrieved. Then
\begin{align*}
\texttt{recall} &= \frac{\text{\# of retrieved relevant documents}}
       {\text{\# of relevant documents}} \\
\texttt{precision} &= \frac{\text{\# of retrieved relevant documents}}
       {\text{\# of retrieved documents}}
\end{align*}

In what follows we discuss the metrics that we have used to measure the
quality of the results of the algorithms introduced here and compare it
with the quality of competitors such as SpaceSaving.
The following random variables will be essential for our definitions below:
\begin{enumerate}
\item The indicator random variables $Y_i$, $1\le i\le n$, telling us whether
  $x_i$ is returned as a result or not, that is, $Y_i=1$ if $x_i$ is returned as
  a result of the algorithm (as a potential top-$k$ most frequent or a
  heavy hitter), and $y_i=0$ otherwise.
\item For top-$k$ queries, $R_i$ will denote
  the \emph{rank} of the element that
  the algorithm reports as $i$-th most frequent. For any element $x$,
  $\text{rank}(x)$ is defined as 1 plus the number of elements in the stream
  with frequency strictly larger than $x$. Thus, if all frequencies
  were distinct, $\text{rank}(x_i)=i$, for all $i$, $1\le i\le n$.
  We also introduce
  $\rho_i$ and $F_i$, the actual relative and absolute frequency, respetively,
  of the element reported as the $i$-th
  most frequent by the algorithm. Notice that $\rho_i = p_{R_i}$, and
  $F_i = f_{R_i}$.
\item For any element $x_i$, we denote $f'_i$ the estimated absolute frequency
  of $x_i$ (by convention, we set $f'_i=0$ if $x_i$ is not in the sample).
  Likewise, $p'_i=f'_i/N$ will denote the estimated relative
  frequency of $x_i$.
\end{enumerate}

\subsection{Recall \& Precision}
In the case of top-$k$ queries, since the number of returned elements and
the number of relevant elements are both equal to $k$, recall and precision
are identical. We will hence only work with recall (or precision, whatever
name serves):
\[
R_u = \frac{Y_1+Y_2+\cdots+Y_k}{k}.
\]
Indeed, the top-$k$ most frequent (relevant) elements are $x_1$, $x_2$,
\ldots, $x_k$ and $Y_1+Y_2+\cdots+Y_k$ is the number of returned elements which are among the top-$k$ most frequent.

One drawback of this metric is that it penalizes in exactly the same amount the
miss of $x_i$, $1\le i\le k$; however, it is reasonable that we consider as
a better algorithm one that misses $x_k$ (the $k$-th most frequent element)
than one that missed $x_1$ (the most frequent element). Hence we introduce
weighted recall:
\[
R = \frac{p_1Y_1+\cdots+p_kY_k}{p_1+\cdots+p_k};
\]
our first metric $R_u$ will be called \emph{unweighted recall}.

\bigskip

In heavy hitter queries with threshold $c$, $0 < c < 1$,
the number of returned elements and the
number of relevant elements are different. The first is given by
\[
Y_1+\cdots+Y_n,
\]
whereas the second is $k^\ast$, the largest index such that $p_{k^\ast} \ge c$.
Hence the recall metrics (unweighted and weighted) are
\begin{align*}
  R_u &= \frac{Y_1+\cdots+Y_{k^\ast}}{k^\ast} \\
  R   &= \frac{p_1Y_1+\cdots+p_{k^\ast}Y_{k^\ast}}{p_1+\cdots+p_{k^\ast}}
\end{align*}
The (unweighted) precision is
\[
P_u = \frac{Y_1+\cdots+Y_{k^\ast}}{Y_1+\cdots+Y_n},
\]
and a weighted precision can also be introduced as
\[
P  = \frac{p_1Y_1+\cdots+p_kY_{k^\ast}}{p_1Y_1+\cdots+p_nY_n}.
\]

Notice that in all cases the metrics adopt values in $[0,1]$; lower
values indicate poorer quality of the returned results, with relevant
elements missing in the result and/or irrelevant elements (not among
the top-$k$ most frequent, not heavy hitters) reported as
``relevant''. Values near 1 correspond to very good approximations to
the exact answer. Take for instance $P$. Since
\[
p_1Y_1+\cdots+p_kY_{k^\ast} \le p_1Y_1+\cdots+p_nY_n,
\]
it is clear that its value will be always $P\le 1$. If no heavy hitter
were reported then $p_1Y_1+\cdots+p_kY_{k^\ast}=0$ and $P=0$. Suppose
that some algorithm $A$ reports $k'$ elements, of which
only $k''\le k^\ast$ are heavy hitters, say
$x_{i_1}, \ldots,
x_{i_{k''}}$, with $1\le i_1\le i_2\le i_{k''}\le k^\ast$. Then if 
\[
v = p_{i_|}+\cdots+p_{i_{k''}}
\]
we will have
\[
P = \frac{v}{v+w},
\]
where $w\ge 0$ is the sum of the relative frequencies of the other $k'-k''$
returned elements which are not heavy hitters. If $k'=k''$ then $w=0$
and the precision is 1: all reported elements are heavy hitters. However, the
recall $R$ might be less than 1 as some heavy hitters might have
not been reported, i.e., $k''\not= k^\ast$.

We also remark here that in heavy hitter queries, since all relevant elements
have relative frequency $\ge c$ by definition, there will be many
applications where all elements will be considered equally relevant. Hence,
the weighted recall and weighted precision metrics might be of lesser interest
in those cases.

\subsection{Precision-like Metrics}
\begin{comment}
  La m�trica que has venido denominando ``discrete precision'' es el
  \emph{unweighted recall} explicado en el apartado anterior? En caso
  contrario, qu� otras m�tricas has usado en los estudios de top-$k$
  most frequent? Completar esta subsecci�n o eliminarla, seg�n proceda.
\end{comment}

\subsection{Order}
In the top-$k$ most frequent problem, it might be often the case that
the order in which elements are reported matters. So we will have metrics to
measure how different is the real rank $R_i$ of the $i$-th reported element
from the pretended rank ($i$).
Notice that the errors in the ordering typically come from bad estimates of
the actual frequencies of the elements. Most algorithms top-$k$
will keep a ``sample'' of $m \le k$ elemnts from the stream,
gather statistics for the sampled elements, and report
the first $k$ elements of the sample when sorted
in non-increasing order of \textbf{estimated}
frequencies. But it makes sense to consider metrics which take
into account only the order induced by the estimated frequencies and not
the estimated frequencies themselves (see the subsection below).
For instance an algorithm might
systematically and grossly overestimate the real frequencies of
the sample elements, but these estimates might ``preserve'' well enough
the real relative ordering of the elements. 

\begin{comment}
  Ninguna de estas medidas esta normalizada. Habra que estudiar cuales son
  ``relevantes'' (tienden a 0 en algoritmos malos, tienden a
  1 en algoritmos buenos, discriminan entre algoritmos diferentes),
  y cuales no. Hay que agregar tambien la variante que propones.
  A continuacion, las metricas, a palo seco.
\end{comment}

\begin{align}
  e &= \sum_{1\le i\le k} p_i R_i
  e' &= \sum_{1\le i\le k} p_i \left(1-p_{R_i}\right)
  IC_u &= \sum_{1\le i\le k} (R_i-i)^2
  IC &= \sum_{1\le i\le k} p_i (R_i-i)^2
\end{align}


\subsection{Estimated frequencies}
Finally we consider
metrics which measure the quality of the estimated frequencies. We discuss
first these metrics in the context of top-$k$ most frequent queries.

Error type I $\epsilon_1$ measures the deviation of the estimated
frequencies for the frequent elements:
\[
\epsilon_I = \frac{\sum_{1\le i\le k} (f_i-f'_i)^2}{\sum_{1\le i\le k} f_i^2}.
\]
For reasons that will be discussed later we might also take
$f'_i:=\min\{f'_i,2f_\text{obs}(x_i)\}$. Since $f_\text{obs}(x_i)\le f_i$
for all $i$, the numerator in $\epsilon_I$ cannot exceed the denominator
and thus $\epsilon_I$ will range between 0 (perfect estimation and all frequent
elements in the sample) and 1 (all frequent elements missing and/or
grossly overestimated). This meaure does not take into account what are
the non-frequent elements that are reported, or how good or bad are the
estimates of their corresponding frequencies. For that we have a second measure,
error type II. In this case the absolute errors in the estimates are
not that useful; we should penalize more severely the errors in the
estimation of elements of very low frequency since that might
imply reporting them as frequent elements and they are not even close. On the
other hand, no penalty should be added if the element is not sample.
Hence for the definition of error type II we will be doing better using relative
errors, and weighting respect their presence or not in the result:
\[
\epsilon_{II}=\frac{1}{k}
\sum_{i\ge k} Y_i\left(\frac{\hat{f}_i-f_i}{f_i}\right)^2.
\]
Since each term in the sum above is at most 1 and there are at most $m$
non-null terms, $\epsilon_{II}$ also ranges between 0 (no mistakes in the
frequency estimation) and 1 (all sampled items are infrequent and we make
a gross overestimation of their frequencies).

\begin{comment}
  Revisa las definiciones para ver si son correctas y se adecuan
  con las que usas
  en los experimentos. Agrega nuevas medidas que estes usando y se te ocurran.
  Ya habra ocasion de ``purgar'' aquellas que resulten poco relevantes,
  redundantes o demasiado ad-hoc (como comentamos tiene que haber
  un fundamento racional que justifique la definicion de la medida, no
  vale que de los resultados que nos gustan :-))
  
  $R_u$ y $P_u$ las versiones no ponderadas de \emph{recall} y \emph{precision}
  no son las mas utiles, pero han de estar y aparecer los valores
  correspondientes en la parte experimental. En el paper de SS, son las medidas
  que se usan y estara bien hacer la comparativa en base a estas dos
  medidas tambien.
\end{comment}


\section{Analysis of the Algorithms}

\section{Experimental Results}
\begin{comment}
  Esta secci�n (que ocupar� una parte importante de este documento y
  tambi�n en tu TFG) se dividira en no menos de tres partes:
  \begin{itemize}
  \item Descripci�n de los conjuntos de datos reales y sint�ticos usados
    en los experimentos
  \item Descripci�n del \emph{setup} experimental: cuantas veces
    se repetia el experimento, entorno de trabajo, etc.  
  \item Presentaci�n (tablas, gr�ficas) y an�lisis de los resultados
    experimentales sobre calidad de los algoritmos (recall, precision, errores
    de estimacion de frecuencias, \ldots)
  \item Presentaci�n (tablas, gr�ficas) y an�lisis de los resultados
    experimentales sobre rendimiento (tiempo de proceso, memoria,\ldots)
  \end{itemize}
\end{comment}
\end{document}

%&LaTeX   -*-LaTeX-*-
% $Id$

\documentclass{article}
\usepackage{amsmath,amssymb,amsfonts}
\usepackage{mathbbol}
\usepackage[latin1]{inputenc}
\usepackage{float}
\usepackage{graphicx}
%\usepackage{algorithms}
\usepackage[noend]{algpseudocode}
\usepackage{cite}
\usepackage{hyperref}
\usepackage{tikz}

\usepackage{fancybox}

\graphicspath{{figs/}}

\newcommand{\bigOh}{\mathop{\mathcal{O}}\nolimits}  % big O
\newcommand{\esimo}[1]{\text{${#1}$-th}}              %n-th
\newcommand{\node}[2]{\langle {#1},{#2}\rangle}
\newcommand{\Prob}[1]{\Pr\left\{{#1}\right\}}
\newcommand{\Exp}[1]{\mathop{\mathbb{E}}\left\{{#1}\right\}}
\newcommand{\Var}[1]{\mathop{\mathbb{V}}\left\{{#1}\right\}}
\newcommand{\email}[1]{E-mail: \texttt{#1}}
\newcommand{\hypergeom}[6]{{}_{#1}{#2}_{#3}           %
\left(\left.\genfrac{}{}{0pt}{}{#4}{#5}\,\right|\, #6\right)} % hipergeometrica
\newcommand{\Ind}[1]{\mathbb{1}_{{#1}}}
\newcommand{\eqlaw}{\stackrel{\mathcal{D}}{=}} 
\newcommand{\C}{\mathcal{C}}
\newcommand{\U}{\mathcal{U}}
\newcommand{\K}{\mathcal{K}}



\newtheorem{theorem}{Theorem}
\newtheorem{defn}{Definition}

\renewcommand\algorithmiccomment[1]{\(\qquad\triangleright\) #1}%
\newcommand{\Proc}[1]{\text{\textsc{#1}}}
\newcommand{\To}{\textbf{\ to\ }}
\newcommand{\swap}{\leftrightarrow}
\newcommand{\call}[2]{\textsc{#1}({#2})}
\floatstyle{ruled}
\newfloat{algorithm}{thp}{loa}
\floatname{algorithm}{Algorithm}

\hypersetup{%
colorlinks = true,
breaklinks = true,
linkcolor = blue,
anchorcolor = blue,
citecolor = magenta,
pagecolor = blue,
urlcolor = red,
filecolor = blue,
bookmarksopen = true,
bookmarksnumbered = true,
backref = true,
}

\newsavebox{\mybox}
\newlength{\mylength}

\newenvironment{comment}[1][\linewidth]{
\setlength{\mylength}{#1}
\begin{lrbox}{\mybox}\begin{minipage}{\mylength}
\color{red}}{
\end{minipage}\end{lrbox}
\bigskip\centerline{\shadowbox{\usebox{\mybox}}}}

\title{Lottery Sampling, Lottery Saving and other Algorithms
  for Top $k$ Frequent Elements}

\author{Conrado Mart{\'\i}nez \and Gonzalo Solera} %[2]

\date{\today}

\begin{document}
\maketitle
\begin{comment}
  Es muy buena idea que vayas escribiendo en este documento de trabajo
  u otro equivalente que constituya la base de tu documentaci�n del
  TFG, las diferentes cosas que vayas haciendo, de la manera m�s
  organizada que puedas, aunque no es� muy pulido. Este documento te
  da un buen punto de partida sobre c�mo organizar tu TFG (habr�s
  de incluir algunas secciones ``chorras'', you know it) y tambi�n
  va estableciendo el marco de trabajo (p.e. esquema de nombres
  para los diferentes algoritmos, la plantilla ``unificada'' para todos ellos,
  las definiciones de las medidas de rendimiento, \ldots)
\end{comment}

\section{Preliminaries}
Consider a (large) data stream $\mathcal{Z}=z_1,\ldots,z_N$, where each $z_j$ is
drawn from some domain or \emph{universe} $\mathcal{U}$, and let $n\le N$ 
be the number of distinct elements in $\mathcal{Z}$. We may thus look at 
the multiset $X$ underlying $\mathcal{Z}$
\[
X = \{x_1^{f_1},\ldots,x_n^{f_n}\}
\]
where $x_1$, \ldots, $x_n$ are the $n$ distinct elements that occur 
in $\mathcal{Z}$ and $f_i$ denotes the number of occurrences (absolute 
frequency) of $x_i$ in $\mathcal{Z}$.
We will assume, w.l.o.g., 
that we index the elements in $X$ in non-increasing order
of frequency, thus $f_1\ge f_2 \ge \cdots \ge f_{n-1} \ge f_n > 0$.
We will use $p_i = f_i/N$ to denote the relative frequency of $x_i$.
For simplicity, we will assume that $f_1 > f_2 > \cdots > f_n$ in the
definitions below---they can be more or less easily adapted to cope
with elements
of identical frequency.

The two problems that we want to study here are:
\begin{enumerate}
\item Top $k$ most frequent elements. Given $\mathcal{Z}$ and a value
$k\le n$, we want to find 
$\{x_1,\ldots,x_k\}$ (or any subset of $k$ distinct elements
  with maximal frequencies). 
\item Heavy hitters. Given $\mathcal{Z}$ and a value $c$, $0 < c < 1$, 
we want to find (or count) the number
of distinct elements in $\mathcal{Z}$ with relative frequency $p_i \ge c$.
Those elements are called \emph{heavy hitters}. 
Given the data stream and the value $c$, we want to obtain
$\{x_1,\ldots,x_{k^\ast}\}$, where $k^\ast$ is
the largest value $k$ such that $p_k \ge c$. The value $k^\ast$
is the number of heavy hitters. 
\end{enumerate}
Moreover, in both problems, we might want that the algorithm
  returns the frequency $f_i$ of the returned elements. 

  None of these two problems can be solved exactly unless we can keep
  $\Theta(n)$ elements in memory; thus under the tight memory
  constraints of the data stream model, we must aim at approximate
  good solutions. Hence, the algorithms that we describe next might
  return elements which are not among the most frequent elements, or
  that are not heavy hitters, and rather than the frequencies $f_i$ of
  the returned elements, we will have to content ourselves with
  estimations $f'_i$ of the real frequencies.

We will concentrate in algorithms for
top $k$ most frequent elements. Notice that there can be at most
$\lceil 1/c\rceil$ heavy hitters in a data stream, and thus an algorithm
that retrieves the top $k^\ast=\lfloor 1/c\rfloor$ most frequent elements will
obtain all the heavy hitters.


\section{The Algorithms}
The algorithms that we shall consider next fail under the following
scheme.  They all keep a \emph{sample} $\mathcal{S}$ of up to $m$
distinct elements, for some value $m\ge k$ fixed in advance. Each element
$x$ in the sample is eqquiped with a frequency counter
\[
f'(x) = f_\text{obs}(x) + f_\text{ini}(x),
\]
which is the sum of two components: $f_\text{obs}(x)$ is the number of
times that $x$ has been observed in the data stream since it has been included
in $\mathcal{S}$; $f_\text{ini}(x)$ is an estimation of the number of
occurrences of $x$ prior to the occurrence in which $x$ has been added
to the sample.

In the initial phase, the algorithms populate the sample with the
initial distinct elements in the data stream, recording their
frequencies. In particular, $f_\text{ini}(x)=0$ for these elements,
and $f'(x)$ will be the real frequenciy of $x$ in the ``prefix'' of
the data stream examined so far.  This first phase ends when $m$
distinct elements have already been collected and an \esimo{(m+1)}
distinct element occurs in the data stream, or when the data stream is
exhausted---because $m\ge n$.  In the latter case, the sample has
complete information about the full data stream and all required
information can be obtained exactly.  In the interesting case, when $m
< n$ (typically, $m\ll n$), the algorithms loop through the remaining
items in the data stream $\mathcal{Z}$. For every incoming item $z$,
if $z$ is already in $\mathcal{S}$, the algorithms update $f'(z)$ and
perhaps perform some additional bookkeeping (depending on the
particular algorithm under consideration), but not much more has to be
done. On the other hand, if $z\not\in\mathcal{S}$ then we apply some
criterium $\texttt{add?}(z,\mathcal{S})$;
if $\texttt{add?}(z,\mathcal{S})$ is not satisfied,
then $z$ is discarded. Otherwise, some element $z'\in\mathcal{S}$ is choosen
and evicted from the sample ($z'=\texttt{element\_to\_evict}(\mathcal{S})$),
and $z$ is added to $\mathcal{S}$; the frequency counter is initialized
$f'(z)=1+f_\text{ini}(z)$. The algorithms that we will consider differ
hence in three main aspects:  
\begin{enumerate}
\item When do we add to the sample an incoming item $z\not\in\mathcal{S}$
  ($\texttt{add?}(z,\mathcal{S})$? In several algorithms that we will
  consider later, this is a randomized choice.
\item Which element $z^\ast$ do we evict from the sample when a new element
  $z$ has to be added to the sample
  ($\texttt{element\_to\_evict}(\mathcal{S})$)?
  Again, this choice might be probabilistic.
\item How do we estimate the frequency $f_\text{ini}(z)$, that is, the number
  of prior occurrences of a new element $z$ that is to be added to the sample?
\end{enumerate}

Take for instance the well-known \textsc{SpaceSaving} (SS) \cite{SpaceSaving}.
It always
adds incoming items $z\not\in\mathcal{S}$, that is,
$\texttt{add?}(z,\mathcal{S})=\textbf{true}$. The element $z^\ast$ selected
for eviction is one with lowest estimated frequency
\[
z^\ast = z_\text{lef} = \text{arg min}_{x\in\mathcal{S}}\{f'(x)\}.
\]
And the newly added item $z$ inherits the frequency of the evicted item:
$f_\text{ini}(z)=f'(z^\ast)$.

Let us know consider our first new algorithm \textsc{LotterySampling} (LS).
Each element $x$ in the sample has, besides its frequency counter, a ticket
$t(x)\in[0,1)$. When a new item $z$ in the data stream is retrieved
LS generates, uniformly at random in $(0,1)$, a ticket $t$ for $z$. If $z$
was already in the sample its frequency counter $f'(z)$ is updated (as usual),
but also its ticket: if $t > t(z)$ then $t(z):= t$. If $z$ is not
in the sample, we compare $t$ with the minimum ticket
$t_\text{min}=t(z_\text{min})$
in the sample. If $t > t_\text{min}$ then $z$ is added to the sample,
otherwise $z$ is discarded. The element $z^\ast$ to be evicted
is $z_\text{min}$, the one with the
minimum ticket, and the frequency counter of $z$ is initialized with
\[
f_\text{ini}(z)=\left\lfloor\frac{1}{1-t_\text{min}}\right\rfloor.
\]

\textsc{LotterySaving-LowestEstimatedFrequency} (LS-LEF) combines
some of the ideas of the two
algorithms above. When an item $z$ is not in the sample we compare its
ticket $t$ with the ticket $t_\text{lef}=t(z_\text{lef})$ of an element
$z_\text{lef}$
with smallest frequency counter (in cae of ties, any of them is picked
at random). Notice that $t_\text{lef}$
might be or not the minimum
ticket in the sample. If $t > t_\text{lef}$ then $z$ is added
and $z^\ast=z_\text{lef}$ is evicted, and
$f_\text{ini}(z)=f'(z^\ast)$.

\textsc{LotterySaving-LeastRecentlyObserved} (LS-LRO) is similar to LS-LEF,
but the ticket
$t$ a candidate element $z$ to enter the sample is compared with the
ticket $t_\text{lro}=t(z_\text{lro})$ of the item $z_\text{lro}$
in the sample that was observed least recently (i.e., 
$z_\text{lro}$ is the item with frequency counter updated the longest ago). 
Like in LS-LEF, if $t > t_\text{lro}$ then $z$ is added and
$z^\ast=z_\text{lro}$ is evicted. The counter of $z$ is initialized using
$f_\text{ini}(z)=f'(z_\text{lro})$.

In \textsc{LotterySaving-SmallestTicket} (LS-ST) the element
to be evicted is $z_\text{min}$; but instead of the rule for initialization of $f'$ of LS it uses the estimated frequency $f'$ of $z^\ast$,
like LS-LEF and LS-LRO. That is, when an item $z$ is not in the sample
we compare its
ticket $t$ with the smallest ticket $t_\text{min}=t(z_\text{min})$ of the
element $z_\text{min}$
with smallest tickets (w.l.o.g. we can safely assume that no two tickets
ever are identical).
If $t > t_\text{min}$ then $z$ is added
and $z^\ast=z_\text{min}$ is evicted, like in LS. But, unlike LS,
we have $f_\text{ini}(z)=f'(z^\ast)$.


\textsc{LotterySaving-Threshold} (LS-THR-$\theta$) has a parameter
$\theta\in[0,1]$.
We omit the parameter $\theta$ in the name, thus write LS-THR in
generic discussions about
this algorithm.
A new item $z\not\in\mathcal{S}$ is added to the sample if and only if
the ticket $t$ generated for $z$ is larger that $\theta\cdot t_\text{max}$,
where $t_\text{max}$ is the largest ticket among the elements in the sample
(LS-THR-0 is equivalent to SS).

If $z$ is added to $\mathcal{S}$, LS-THR evicts the element $z^\ast=z_\text{lef}$
with minimum frequency counter and $z$ inherits the frequency counter
of $z^\ast$ as the initial estimation: $f_\text{ini}(z)=f'(z^\ast)$.

\textsc{LotterySaving-AboveMean} (LS-AM) adds a  new item to the sample
if its ticket $t$ is larger that the mean value $\overline{t}$ of all
the tickets in the sample. When $t > \overline{t}$, the evicted item $z^\ast$
is $z_\text{lef}$, one with smallest frequency counter,
and $z$ inherits $f'(z^\ast)$ like
in LS-LEF and LS-THR.

Another variation is \textsc{LotterySaving-AboveMedian} (LS-MED) which adds
a new item to the sample if its ticket $t$ is above the median $t_\text{med}$
of the tickets in the sample. The remaining choices for the algorithm are as
in LS-AM. LS-MED can be generalized to consider the $\alpha$-quantile
(LS-QUANT-$\alpha$) of the tickets in the sample. Notice that LS-MED is 
LS-QUANT-0.5; the algorithm uses $t_{(\lceil \alpha\cdot m\rceil)}$ to decide
whether to include or not a new item in the sample, where $t_{(r)}$
denotes the $\esimo{r}$ smallest ticket in the sample, $1\le r\le m$.

The basic variants of LS-LEF and LS-LRO can be modified to incorporate
the ideas behind LS-AM and LS-MED.
Thus LS-LEF-AM and LS-LRO-AM will add an element
$z\not\in\mathcal{S}$ to the sample if $t > t(z^\ast)$
\textbf{or} $t$ is above the mean value $\overline{t}$ of the tickets
in the sample. The evicted item $z^\ast$ and the initialization of
the frequency counter of $z$ is as in the corresponding basic algorithms.

Instead of the mean value $\overline{t}$ of the tickets in the sample,
we might consider the median of the tickets, or more generlly, the
$\alpha$-quantile of the tickets, getting the variants LS-LEF-MED
LS-LRO-MED, or more generally, LS-LEF-QUANT-$\alpha$ and LS-LRO-QUANT-$\alpha$. 

Notice that it makes no sense to ``combine''
LS-AM and LS-QUANT with LS-ST since $t_\text{min}<\overline{t}$ and
$t_\text{min}\le t_{(r)}$ for all $r$.

Another source of variations stems from the way tickets are updated
when an item $z$ already in the sample is the incoming element from
the stream.  In all cases above we have assumed that tickets are
updated via $t(z):= \max(t, t(z))$, where $t$ is the ticket generated
for the current instance of $z$, whereas $t(z)$ is the ticket
associated to $z$ in the sample.  We can derive corresponding variants
of all the others discussed above were the items retain the ticket $t$
with which they entered the sample, and never update it once in the
sample, for instance.

All algorithms using tickets can be also characterized as defining a
\emph{thresold value} $t^\ast$ and a candidate item $z^\ast$ for eviction. The
criterium for addition of a element $z\not\in\mathcal{S}$ is always whether
$t > t^\ast$ or not, where $t$ is the ticket that was generated for
the current instance.
For example, lottery sampling (LS) takes
$t^\ast\equiv t_\text{min} = \min\{t(z)\,|\,z\in\mathcal{S}\}$,
and $z^\ast$ an element with the
minimum ticket $t_\text{min}$. Recall that we can safely assume that
all tickets in $\mathcal{S}$ are distinct.

With these conventions the template for all the algorithms discussed above is
as follows:
\begin{verbatim}
Populate S with the first m distinct elements in Z, 
         updating tickets and frequencies
while Z is not exhausted do
      z:= current item in Z
      t:= Random(0,1)
      if z in S then
           Update the ticket t(z)
           f'(z):= f'(z) + 1
      else if t > t* then
           S:= S - {z*} + {z}
           t(z):= t
           f'(z):= fini(z) + 1
      else // do nothing
      Update t* and z*
endwhile
Report the k elements in S with largest f'
\end{verbatim}

Table \ref{table:algorithms} summarizes the characteristic values of
$t^\ast$ and $z^\ast$. Observe that we have considered two options for
the initialization of $f_\text{ini}(z)$ and three options for $z^\ast$
($z_\text{min}=\text{the element with minimum $t$}$,
$z_\text{lef}=\text{an
element with minimum $f'$}$, and $z_\text{lro}=\text{the
least recently observed element}$); if we contemplate $C$ different ways
to define $t^\ast$ ($C=9$ in our table), we could consider then up to $6C$
different possible combinations---some make no sense.
If we add on top of that the choice
to update or not tickets of sampled elements we raise the count to
$12C$. Moreover, LS-THR and LS-QUANT depend on an additional real parameter
in $[0,1]$ giving us an additional source of variability.

\begin{table}
  \begin{tabular}{llll}
    Algorithm & $t^\ast$ & $z^\ast$ & $f_ini(z)$ \\\hline\hline
    SS        & 0       & $z_\text{lef}$ & $f'(z^\ast)$ \\
    LS        & $t_\text{min}=t(z^\ast)$ & $z_\text{min}$
    & $\lfloor\frac{1}{1-t^\ast}\rfloor$ \\
    LS-LEF    & $t(z^\ast)$ & $z_\text{lef}$ & $f'(z^\ast)$ \\
    LS-LRO    & $t(z^\ast)$ & $z_\text{lro}$ & $f'(z^\ast)$ \\
    LS-ST    & $t(z^\ast)$ & $z_\text{min}$ & $f'(z^\ast)$ \\
    LS-THR    & $\theta\cdot t_{(m)}=\theta\cdot t_\text{max}$ & $z_\text{lef}$ & $f'(z^\ast)$ \\
    LS-AM     & $\overline{t}=\sum_{z\in\mathcal{S}}t(z)/m$ & $z_\text{lef}$ & $f'(z^\ast)$ \\
    LS-QUANT  & $t_{\lceil\alpha m\rceil}$ & $z_\text{lef}$ & $f'(z^\ast)$ \\
    LS-LEF-AM & $\min\{\overline{t},t(z^\ast)\}$ & $z_\text{lef}$
    & $f'(z^\ast)$ \\
    LS-LEF-MED & $\min\{t_{\lfloor(m+1)/2\rfloor},t(z^\ast)\}$ &
    $z_\text{lef}$ & $f'(z^\ast)$
  \end{tabular}
\end{table}

The three strategies LS-ST, LS-LEF and LS-LRO are almost identical, they
only differ in the definition of the element $z^\ast$ to be evicted.
These three strategies will be called \emph{canonical}, as all the
other algorithms that we have discussed here can be seen as variantions of one
of the three; maybe in the way that tickets are updated, maybe in the way
that $t^\ast$ is defined (e.g., $t^\ast\not=t(z^\ast)$), or the way that
$f'$ is initialized (e.g., $f_ini(z)\not=f'(z^\ast)$).

It is clear that the combinations that we can make ``playing'' around
with the criteria for addition ($\texttt{add?}(z,\mathcal{S})$), the
criteria for eviction and the initialiation of the frequency counter
are almost endless, but here we will restrict ourselves to the those
that we have been able to analyze and for which the experiments give
the best results. Another important issue in our choice has been the
efficiency with which the different operations can be supported.
Thus for instance, to implement LS-LEF it is very convenient to
maintain the elements of $\mathcal{S}$ sorted by frequency counter; locating
the least frequent element becomes trivial, and maintaining the order of the
sample after each frequency update is also very easy and efficient, as
they increment one by one.

\begin{comment}
  Gonzalo: agrega a esta seccion otros algoritmos que estes probando.
  En el documento final es posible que hayas de hacer una ``purga'',
  eliminando las opciones que no han dado buenos resultados. Sin
  embargo, SpaceSaving, Lottery Sampling y las tres variantes
  canonicas LS-ST, LS-LEF y LS-LRO deben estar siempre, como
  referencia. Para nuevas variantes inventa nombres y acronimos. Tambien
  habriamos de pensar algun modo facil de indicar si una estrategia es con o sin
  update de tickets, en vez de inventar nuevos nombres. Por ejemplo, LS*-LEF es
  la variante de LS-LEF sin hacer update de los tickets.

  Por cierto que nn una primera impresion yo
  diria que LS*-LEF es casi lo mismo que
  SpaceSaving pero lanzando una moneda e incluyendo al item z nuevo en
  el sample con probabilidad $m/(m+1)$ (con probabilidad $1/(m+1)$ el item
  nuevo no reemplaza a $z_\text{lef}$). 
\end{comment}

\subsection{Implementing the Algorithms}

\section{Measures of Quality}
Of course, for a top $k$ frequent elements algorithm we would like
that it reports the true $k$ most frequent elements and their
respective frequencies, but we know that this will not be possible unless
we had a linear amount of memory ($Theta(n)$)---this includes
the trivial situation when $m\ge n$. Hence we will assume in waht follows
that $m\ll n$, and we would like that
our algorithms return good approximations: elements that are
the most frequent or close to that, and good estimations of their respective
frequencies. 
To measure the quality of the different algorithm
we introduce several measures, in which
two random variables will be essential:
\begin{enumerate}
\item The indicator variable $Y'_i$ tells us whether $x_i$
  has been sampled or not: $Y'_i=1$
  if $x_i\in\mathcal{S}$ and $Y'_i=0$, otherwise. Notice that the $Y'_i$'s
  are not independent and $Y'_1+\cdots+Y'_n=m$.
\item Since $m\ge k$, the algorithm has to choose $k$ out of the $m$
  elements to report them as the top $k$ most frequent elements.
  Quite naturally, in all our algorithms the $k$ elements with largest
  frequency counter will be the ones to be reported and their estimated
  frequencies will be those collected by the algorithm. The indicator
  variable $Y_i=1$ if $x_i$ is reported, $Y_i=0$ otherwise
  (either because $x_i\not\in\mathcal{S}$ or because
  $f'(x_i)$ is not among the largest
  $k$ values of $f'$ in $\mathcal{S}$). Like the $Y'_i$'s, the $Y_i$'s
  are not independent; and $Y_1+\cdots+Y_n=k$.
\end{enumerate}

Our first measure, \emph{(weighted) recall}, measures the proportion of
top frequent elements actually found by the algorithm:
\[
R = \frac{p_1 Y_1+\cdots+p_k Y_k}{p_1+\cdots+p_k}
\]
A proxy for $R$ is given by
\[
R' = \frac{p_1 Y'_1+\cdots+p_k Y'_k}{p_1+\cdots+p_k}, 
\]
which is always an upper boud for $R$, since $Y_i \le Y'_i$ for all $i$,
$1\le i\le n$.
Notice that, in all cases, $0\le R\le 1$: $R=0$ if no true frequent element
is reported, and $R=1$ if exactly the top $k$ most frequent elements
are reported. The weights $p_i$ will penalize more algorithms that miss
the most frequent elements among the top frequent, and will lessen the
impact in $R$ if the algorithm missed less frequent elements
(albeit among the most frequent). The same is true for $R'$.

Another measure is \emph{(weighted) precision}, which gives us an indication
of the quality of the reported elements:
\[
P = \frac{p_1 Y_1+\cdots+p_n Y_n}{p_1+\cdots+p_k}
\]
The numerator sums the weights of all reported elements; since exactly
$k$ elements are reported $0 < P\le 1$. The proxy $P'$ for $P$
needs a different ``normalizing'' denominator:
\[
P' = \frac{p_1 Y'_1+\cdots+p_n Y'_n}{p_1+\cdots+p_m}
\]

We can also use the ``ordinary'' measures of unweighted recall $R_u$ and
unweighted precision $P_u$ used in most of the literatire (e.g. in the
experimental study of SS \cite{}). In terms of our indicator random iables
we have
\begin{align*}
  R_u &= \frac{Y_1+\ldots+Y_k}{k} \\
  P_u &= \frac{Y_1+\ldots+Y_k}{m},
\end{align*}
and the approximations (actually, upper bounds)
\begin{align*}
  R'_u &= \frac{Y'_1+\ldots+Y'_k}{k} \\
  P'_u &= \frac{Y'_1+\ldots+Y'_k}{m}.
\end{align*}

Besides the measures of ``recall'' and ``precision'' we also consider
measures for the quality of the estimated frequencies.

Error type I $\epsilon_1$ measures the deviation of the estimated
frequencies for the frequent elements:
\[
\epsilon_I = \frac{\sum_{1\le i\le k} (f_i-\hat{f}_i)^2}{\sum_{1\le i\le k} f_i^2},
\]
where $\hat{f}_i=f'(x_i)$ if $x_i\in\mathcal{S}$ and $\hat{f}_i=0$, otherwise.
For reasons that will be discussed later we might also take
$\hat{f}_i=\min\{f'(x_i),2f_\text{obs}(x_i)\}$. Since $f_\text{obs}(x_i)\le f_i$
for all $i$, the numerator in $\epsilon_I$ cannot exceed the denominator
and thus $\epsilon_I$ will range between 0 (perfect estimation and all frequent
elements in the sample) and 1 (all frequent elements missing and/or
grossly overestimated). This meaure does not take into account what are
the non-frequent elements that get sampel or how good or bad are the
estimates of their frequencies. For that we have a second measure,
error type II. In this case the absolute errors in the estimates are
not that useful; we should penalize more severely the errors in the
estimation of elements of very low frequency since that might
imply reporting them as frequent element and they are not even close. On the
other hand, no penalty should be added if the element is not sample.
Hence for the definition of error type II we will be doing better using relative
errors, and weighting respect their presence or not in the sample:
\[
\epsilon_{II}=\frac{1}{m}
\sum_{i\ge k} Y'_i\left(\frac{\hat{f}_i-f_i}{f_i}\right)^2.
\]
Since each term in the sum above is at most 1 and there are at most $m$
non-null terms, $\epsilon_{II}$ also ranges between 0 (no mistakes in the
frequency estimation) and 1 (all sampled items are infrequent and we make
a gross overestimation of their frequencies).

\begin{comment}
  Revisa las definiciones para ver si son correctas y se adecuan
  con las que usas
  en los experimentos. Agrega nuevas medidas que estes usando y se te ocurran.
  Ya habra ocasion de ``purgar'' aquellas que resulten poco relevantes,
  redundantes o demasiado ad-hoc (como comentamos tiene que haber
  un fundamento racional que justifique la definicion de la medida, no
  vale que de los resultados que nos gustan :-))
  
  $R_u$ y $P_u$ las versiones no ponderadas de \emph{recall} y \emph{precision}
  no son las mas utiles, pero han de estar y aparecer los valores
  correspondientes en la parte experimental. En el paper de SS, son las medidas
  que se usan y estara bien hacer la comparativa en base a estas dos
  medidas tambien.
\end{comment}

\section{Analysis of the Algorithms}

\section{Experimental Results}
\begin{comment}
  Esta secci�n (que ocupar� una parte importante de este documento y
  tambi�n en tu TFG) se dividira en no menos de tres partes:
  \begin{itemize}
  \item Descripci�n de los conjuntos de datos reales y sint�ticos usados
    en los experimentos
  \item Descripci�n del \emph{setup} experimental: cuantas veces
    se repetia el experimento, entorno de trabajo, etc.  
  \item Presentaci�n (tablas, gr�ficas) y an�lisis de los resultados
    experimentales sobre calidad de los algoritmos (recall, precision, errores
    de estimacion de frecuencias, \ldots)
  \item Presentaci�n (tablas, gr�ficas) y an�lisis de los resultados
    experimentales sobre rendimiento (tiempo de proceso, memoria,\ldots)
  \end{itemize}
\end{comment}
\end{document}
